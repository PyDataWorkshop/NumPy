\documentclass[11pt]{article} % use larger type; default would be 10pt

\usepackage{subfiles}

\usepackage{amsmath}

\usepackage{amssymb}

\usepackage[utf8]{inputenc}
\usepackage{geometry} % to change the page dimensions
\geometry{a4paper} 
\usepackage{framed}
\usepackage{graphicx} 
\usepackage{booktabs} % for much better looking tables
\usepackage{array} % for better arrays (eg matrices) in maths
\usepackage{paralist} % very flexible & customisable lists (eg. enumerate/itemize, etc.)
\usepackage{verbatim} 
\usepackage{subfig} 
%-----------------------------------------------------%
\usepackage{fancyhdr} % This should be set AFTER setting up the page geometry
\pagestyle{fancy} % options: empty , plain , fancy
\renewcommand{\headrulewidth}{0pt} % customise the layout...
\lhead{}\chead{}\rhead{}
\lfoot{}\cfoot{\thepage}\rfoot{}

%%% SECTION TITLE APPEARANCE
\usepackage{sectsty}
%\allsectionsfont{\sffamily\mdseries\upshape}
%--------------------------------------------------------------------------------------%
\usepackage[nottoc,notlof,notlot]{tocbibind} % Put the bibliography in the ToC
\usepackage[titles,subfigure]{tocloft} % Alter the style of the Table of Contents
\renewcommand{\cftsecfont}{\rmfamily\mdseries\upshape}
\renewcommand{\cftsecpagefont}{\rmfamily\mdseries\upshape} % No bold!

%--------------------------------------------------------------------------------------%

\voffset=-1.5cm
\oddsidemargin=0.0cm
\textwidth = 470pt


% http://wiki.quantsoftware.org/index.php?title=Numpy_Tutorial_1



\begin{document}

\large
\tableofcontents
NumPy is the fundamental package for scientific computing with Python. It contains among other things:
\begin{itemize}
\item a powerful N-dimensional array object
\item sophisticated (broadcasting) functions
\item tools for integrating C/C++ and Fortran code
\item useful linear algebra, Fourier transform, and random number capabilities
\item Besides its obvious scientific uses, NumPy can also be used as an efficient multi-dimensional container of generic data.  -Arbitrary data-types can be defined. This allows NumPy to seamlessly and speedily integrate with a wide variety of databases.
\end{itemize}
\subfile{numpy1-intro}
\subfile{numpy2-arrays}
\subfile{numpy3-basicmaths}
\subfile{numpy4-random}
\subfile{numpy5-matrices}

\end{document}
