Creating character arrays (numpy.char)
Note
numpy.char is the preferred alias for numpy.core.defchararray.

core.defchararray.array(obj[, itemsize, ...])	Create a chararray.
core.defchararray.asarray(obj[, itemsize, ...])	Convert the input to a chararray, copying the data only if necessary.
Numerical ranges
arange([start,] stop[, step,][, dtype])	Return evenly spaced values within a given interval.
linspace(start, stop[, num, endpoint, ...])	Return evenly spaced numbers over a specified interval.
logspace(start, stop[, num, endpoint, base, ...])	Return numbers spaced evenly on a log scale.
meshgrid(*xi, **kwargs)	Return coordinate matrices from coordinate vectors.
mgrid	nd_grid instance which returns a dense multi-dimensional “meshgrid”.
ogrid	nd_grid instance which returns an open multi-dimensional “meshgrid”.
Building matrices
diag(v[, k])	Extract a diagonal or construct a diagonal array.
diagflat(v[, k])	Create a two-dimensional array with the flattened input as a diagonal.
tri(N[, M, k, dtype])	An array with ones at and below the given diagonal and zeros elsewhere.
tril(m[, k])	Lower triangle of an array.
triu(m[, k])	Upper triangle of an array.
vander(x[, N, increasing])	Generate a Vandermonde matrix.
The Matrix class
mat(data[, dtype])	Interpret the input as a matrix.
bmat(obj[, ldict, gdict])	Build a matrix object from a string, nested sequence, or array.
