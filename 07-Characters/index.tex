Creating character arrays (numpy.char)
Note
numpy.char is the preferred alias for numpy.core.defchararray.

core.defchararray.array(obj[, itemsize, ...])	Create a chararray.
core.defchararray.asarray(obj[, itemsize, ...])	Convert the input to a chararray, copying the data only if necessary.
String operations

This module provides a set of vectorized string operations for arrays of type numpy.string_ or numpy.unicode_. All of them are based on the string methods in the Python standard library.

%==========================================================================================%
\newpage
\section*{String operations}

add(x1, x2)	Return element-wise string concatenation for two arrays of str or unicode.
multiply(a, i)	Return (a * i), that is string multiple concatenation, element-wise.
mod(a, values)	Return (a % i), that is pre-Python 2.6 string formatting (iterpolation), element-wise for a pair of array_likes of str or unicode.
capitalize(a)	Return a copy of a with only the first character of each element capitalized.
center(a, width[, fillchar])	Return a copy of a with its elements centered in a string of length width.
decode(a[, encoding, errors])	Calls str.decode element-wise.
encode(a[, encoding, errors])	Calls str.encode element-wise.
join(sep, seq)	Return a string which is the concatenation of the strings in the sequence seq.
ljust(a, width[, fillchar])	Return an array with the elements of a left-justified in a string of length width.
lower(a)	Return an array with the elements converted to lowercase.
lstrip(a[, chars])	For each element in a, return a copy with the leading characters removed.
partition(a, sep)	Partition each element in a around sep.
replace(a, old, new[, count])	For each element in a, return a copy of the string with all occurrences of substring old replaced by new.
rjust(a, width[, fillchar])	Return an array with the elements of a right-justified in a string of length width.
rpartition(a, sep)	Partition (split) each element around the right-most separator.
rsplit(a[, sep, maxsplit])	For each element in a, return a list of the words in the string, using sep as the delimiter string.
rstrip(a[, chars])	For each element in a, return a copy with the trailing characters removed.
split(a[, sep, maxsplit])	For each element in a, return a list of the words in the string, using sep as the delimiter string.
splitlines(a[, keepends])	For each element in a, return a list of the lines in the element, breaking at line boundaries.
strip(a[, chars])	For each element in a, return a copy with the leading and trailing characters removed.
swapcase(a)	Return element-wise a copy of the string with uppercase characters converted to lowercase and vice versa.
title(a)	Return element-wise title cased version of string or unicode.
translate(a, table[, deletechars])	For each element in a, return a copy of the string where all characters occurring in the optional argument deletechars are removed, and the remaining characters have been mapped through the given translation table.
upper(a)	Return an array with the elements converted to uppercase.
zfill(a, width)	Return the numeric string left-filled with zeros Calls str.zfill element-wise.
